@techreport{Frede460614,
   author = {Frede, D and Khodabakhshian, M and Malmquist, D},
   institution = {KTH, Mechatronics},
   note = {QC 20111204},
   pages = {55},
   publisher = {KTH Royal Institute of Technology},
   title = {A state-of-the-art survey on vehicular mechatronics focusing on by-wire systems},
   series = {Trita-MMK},
   number = {2010:10},
   year = {2010}
}

@inproceedings{Yu,
author="Yu, Z and Xu, S and Xiong, L and Han, W",
journal="", 
publisher="SAE International",
title="An Integrated-Electro-Hydraulic Brake System for Active Safety", 
booktitle="SAE Technical Paper",
year="2016",
month="04",
volume="",
url="http://dx.doi.org/10.4271/2016-01-1640",
pages="",
abstract="AbstractAn integrated-electro-hydraulic brake system (I-EHB) is presented to fulfill the requirements of active safety. Because I-EHB can control the brake pressure accurately and fast. Furthermore I-EHB is a decoupled system, so it could make the maximum regenerative braking while offers the same brake pedal feeling and also good for ADAS and unmanned driving application. Based on the analysis of current electrohydraulic brake systems, regulation requirements and the requirements for brake system, the operating mode requirements of I-EHB are formed. Furthermore, system topological structure and a conceptual design are proposed. After the selection of key components, the parameter design is accomplished by modeling the system. According to the above-mentioned design method, an I-EHB prototype and test rig is made. Through the test rig, characteristics of the system are tested. Results show that this I-EHB system responded rapidly. Upon the experimental results, increasing pressure response time (T90) of I-EHB is 53% shorter than that of conventional brake system and reducing pressure response time (T10) of I-EHB is 70% shorter than that of conventional brake system. The tracking performance of I-EHB is accurate enough. The RMS (root mean square) of tracking error are 1.99 bar and 1.43 bar according to different aim signals. The experimental results proved that I-EHB fulfills the requirements of active safety.",
number="",
doi="10.4271/2016-01-1640"
} 

@article{ahn2009analysis,
  title={Analysis of a regenerative braking system for hybrid electric vehicles using an electro-mechanical brake},
  author={Ahn, JK and Jung, KH and Kim, DH and Jin, HB and Kim, HS and Hwang, SH},
  journal={International Journal of Automotive Technology},
  volume={10},
  number={2},
  pages={229--234},
  year={2009},
  publisher={Springer}
}

@article{milanes2010electro,
  title={Electro-hydraulic braking system for autonomous vehicles},
  author={Milan{\'e}s, V and Gonz{\'a}lez, C and Naranjo, JE and Onieva, E and De Pedro, T},
  journal={International Journal of Automotive Technology},
  volume={11},
  number={1},
  pages={89--95},
  year={2010},
  publisher={Springer}
}

@inproceedings{laine2007coordination,
  title={Coordination of vehicle motion and energy management control systems for wheel motor driven vehicles},
  author={Laine, L and Fredriksson, J},
  booktitle={Intelligent Vehicles Symposium, 2007 IEEE},
  pages={773--780},
  year={2007},
  organization={IEEE}
}

@article{ifedi2013fault,
  title={Fault-tolerant in-wheel motor topologies for high-performance electric vehicles},
  author={Ifedi, Chukwuma J and Mecrow, Barrie C and Brockway, Simon TM and Boast, Gerard S and Atkinson, Glynn J and Kostic-Perovic, Dragica},
  journal={IEEE Transactions on Industry Applications},
  volume={49},
  number={3},
  pages={1249--1257},
  year={2013},
  publisher={IEEE}
}

@inproceedings{zechang2006research,
  title={Research on electro-hydraulic parallel brake system for electric vehicle},
  author={Zechang, Sun and Qinghe, Liu and Xidong, Liu},
  booktitle={Vehicular Electronics and Safety, 2006. ICVES 2006. IEEE International Conference on},
  pages={376--379},
  year={2006},
  organization={IEEE}
}

@article{shyrokau2013vehicle,
  title={Vehicle dynamics control with energy recuperation based on control allocation for independent wheel motors and brake system},
  author={Shyrokau, Barys and Wang, Danwei and Savitski, Dzmitry and Ivanov, Valentin},
  journal={International Journal of Powertrains},
  volume={2},
  number={2-3},
  pages={153--181},
  year={2013},
  publisher={Inderscience Publishers Ltd}
}


@mastersthesis{mastersthesis,
  author       = {Peter Harwood}, 
  title        = {The title of the work},
  school       = {The school of the thesis},
  year         = 1993,
  address      = {The address of the publisher},
  month        = 7,
  note         = {An optional note}
}

@masterthesis{truongdevelopment,
  title={Development of an active braking controller for brake systems on electric motor driven vehicles},
  author={Truong, B},
  school={KTH Royal Institute of Technology},
  year={2014},
  address={Sweden}
}

@ARTICLE{4475522, 
author={C. Line and C. Manzie and M. C. Good}, 
journal={IEEE Transactions on Control Systems Technology}, 
title={Electromechanical Brake Modeling and Control: From PI to MPC}, 
year={2008}, 
volume={16}, 
number={3}, 
pages={446-457}, 
keywords={brakes;compensation;feedback;force control;friction;predictive control;vehicle dynamics;EMB force control problem;electromechanical brake control;electromechanical brake modeling;feedback linearization;friction compensation;gain scheduling;model predictive control;motor torque;vehicle dynamics controller;Cascaded proportional-integral (PI) control;electromechanical brake (EMB) model;electromechanical brakes (EMBs);feedback linearization;friction compensation;gain scheduling;model predictive control (MPC)}, 
doi={10.1109/TCST.2007.908200}, 
ISSN={1063-6536}, 
month={May},}


@ARTICLE{Xiang, 
author={W. Xiang and P. C. Richardson and C. Zhao and S. Mohammad}, 
journal={IEEE Transactions on Vehicular Technology}, 
title={Automobile Brake-by-Wire Control System Design and Analysis}, 
year={2008}, 
volume={57}, 
number={1}, 
pages={138-145}, 
keywords={automobiles;brakes;control system synthesis;fault tolerance;fuzzy control;road safety;stability;automated highways;automobile brake-by-wire control system design;driver interface;electromechanical brake system;fault tolerance design;fuzzy logic control;mechanical-hydraulic backup;networked control system;vehicle control;vehicle safety;vehicle stability;Brake-by-wired;braking model;fault tolerance;networked control systems;stability control}, 
doi={10.1109/TVT.2007.901895}, 
ISSN={0018-9545}, 
month={Jan},}

@inproceedings{2004-01-2050, 
author="C. Line and C. Manzie and M. Good",  
journal="",  
publisher="SAE International",  
title="Control of an Electromechanical Brake for Automotive Brake-By-Wire Systems with an Adapted Motion Control Architecture",  
booktitle="SAE Technical Paper",
year="2004",  
month="05",  
volume="",  
url="http://dx.doi.org/10.4271/2004-01-2050",  
pages="",  
abstract="A disk brake clamp force controller for electromechanical brakes (EMB) in automotive brake-by-wire systems may be obtained from a standard motion control architecture with cascaded position, speed and current control loops by replacing the outer position control loop with a force control loop. When implemented with proportional, integral and differential (PID) controllers this architecture generally performs well for standard motion control problems, but the EMB control problem is differentiated by a large operating range in which non-linear load disturbances such as friction become significant at high clamp forces of up to 30kN. This paper investigates the feasibility of a cascaded PI control architecture for an EMB with the intention of establishing a baseline standard against which the performance of future control schemes may be compared. Simulation results are presented based on an accepted EMB model.",  
number="",  
doi="10.4271/2004-01-2050"  
} 

@book{MechatronicsBook,
  author = {R Isermann},
  title = {Mechatronic Systems: Fundamentals},
  publisher = {Springer-Verlag},
  year = {2005},
  address   = {London}, 
  edition   = {2},
  city      = {London}, 
  isbn      = {1-85233-930-6},
}

@book{book,
  author    = {No One}, 
  title     = {The title of the work},
  publisher = {The name of the publisher},
  year      = 1993,
  volume    = 4,
  series    = 10,
  address   = {The address},
  edition   = 3,
  month     = 7,
  note      = {An optional note},
  isbn      = {3257227892}
}

@ARTICLE{661149, 
author={R. Isermann}, 
journal={IEEE Transactions on Systems, Man, and Cybernetics - Part A: Systems and Humans}, 
title={On fuzzy logic applications for automatic control, supervision, and fault diagnosis}, 
year={1998}, 
volume={28}, 
number={2}, 
pages={221-235}, 
keywords={control system synthesis;fault diagnosis;fuzzy control;QC;VSC;VSS;adaptive control;approximate reasoning;automatic control;automatic supervision;cascade control;fault diagnosis;feedforward control;fuzzy controller design;fuzzy logic applications;fuzzy-rule-based systems;hybrid classical/fuzzy control systems;process automation;process description;qualitative knowledge;quality control;quantitative knowledge;self-tuning control;vagueness;variable structure control;Adaptive control;Automatic control;Design automation;Fault diagnosis;Fuzzy control;Fuzzy logic;Information processing;Process design;Programmable control;Quality control}, 
doi={10.1109/3468.661149}, 
ISSN={1083-4427}, 
month={Mar},}