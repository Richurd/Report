
\documentclass[a4paper,11pt]{kth-mag}

\usepackage[acronym]{glossaries}
\glsdisablehyper
\usepackage{textcomp}
\usepackage[table]{xcolor}
\usepackage[hidelinks]{hyperref}
\usepackage{todonotes}
\usepackage{lmodern}
\usepackage{amsmath}
\usepackage{amsthm}
\usepackage{url}
\usepackage[swedish,english]{babel}
\usepackage{modifications}
\usepackage{multirow}
\usepackage{textcomp}
\usepackage{listings}
\usepackage{color}
\usepackage{amssymb}
\usepackage{dirtree}
\usepackage{array}
\usepackage{pdfpages}
\usepackage{longtable}
\usepackage[acronym]{glossaries}







%\usepackage{comment} % enables the use of multi-line comments (\ifx \fi) 
%\usepackage{fullpage} % changes the margin
%\usepackage[utf8]{inputenc}
%\usepackage[T1]{fontenc}
%\usepackage{graphicx} %To use pictures


\makeglossaries
\setlength\parindent{0pt}

\extrafloats{1000}
\bibliographystyle{ieeetran}
\title{Title goes here}
\subtitle{Thesis WIP}
\foreigntitle{Title in Swedish goes here}
\author{Richard Odell}
\date{\today}
\blurb{Master's Thesis at ITM\\Supervisor: Lars Svensson \\ Examiner: Lei Feng}
\trita{TRITA xxx yyyy-nn}


\newacronym{rcv}{RCV}{Research Concept Vehicle}
\newacronym{mpc}{MPC}{Model Predictive Controller}
\newacronym{itrl}{ITRL}{Integrated Transport Research Lab}
\newacronym{pi}{PI}{Proportional-Integral}
\newacronym{abs}{ABS}{Anti-lock Braking System}
\newacronym{dc}{DC}{Direct Current}
\newacronym{pid}{PID}{Proportional-Integral-Derivative}
\newacronym{kth}{KTH}{The Royal Institute of Technology}





\begin{document}

\frontmatter
\pagestyle{empty}
\removepagenumbers
\maketitle
\selectlanguage{english}


\clearpage
\begin{abstract}

Write the abstract here.

\end{abstract}
\clearpage
\begin{foreignabstract}{swedish}

Abstract in Swedish goes here.

\end{foreignabstract}
\clearpage

\chapter*{Acknowledgements}

I would like to thank...



\glsaddall
\printglossary
\renewcommand{\glsnamefont}[1]{\textbf{#1}}
\printglossary[type=\acronymtype,style=super,nonumberlist,nopostdot,nogroupskip]


\clearpage
\clearpage
\tableofcontents*
\clearpage
\listoffigures*
\clearpage
\listoftables*
\glsaddall



\mainmatter
\pagestyle{newchap}

%%%%%%%%%%%%%%%%%%%%%%%%%%%%%%%%%%%%%%%

\chapter{Introduction}

This chapter will give an introduction to the thesis. 

\section{Background}

 
The \gls{rcv} at \gls{kth} is a platform for research in vehicle autonomy and vehicle dynamics. For precise autonomous operation, accurate actuation of steering input and wheel torque is critical. The electrical wheel motors of the \gls{rcv} can produce torque for acceleration and braking up to a limit. However, for hard braking maneuvers, regenerative braking is not sufficient and a hydraulic brake system actuated by electric linear actuators is used in addition. In the current configuration, the hydraulic brakes actuate fairly slow (>1s before braking request is met). Thus, a redesign of the control software and the mechanical assembly is needed. The task is challenging due to the non-linear and environment dependent dynamics of the hydraulic system.


\section{Research design/Methodology}
The methodology used for this project will be a case study of how to implement one or more brake controllers, which works best for solving the problem stated together with a comparison of the controllers advantages and disadvantages. The case study will have its focus on qualitative research, as there's extensive work already done within this area. The testing will also have a qualitative orientation in the end to validate the brake model as well as how the controller performs. 

\section{Delimitations}
A lot of aspects affects the behavior of the brakes, and to achieve a properly functioning brake with good reaction time will consume much time. Therefore delimitations must be made in order to make the thesis feasible within the limited time frame. The focus of this thesis will be on making an existing braking system faster in terms of reaction time of reaching the requested braking torque. 

Wheel slip and \gls{abs} are both important factors while designing a modern brake, but this will not be implemented or considered in this thesis due to the fact that there is simply not enough time. Split mu, meaning when there is different friction coefficients acting on the the wheels, for example when one wheel travels over a slip of ice or gravel, will not be considered either. The parts that has been left out will be seen as future work that can be made to further improve the brakes on the vehicle. 
 

\section{Ethics}
Brakes is one of the most vital part to the safety of a vehicle. Although this project handles the brakes while in autonomous drive mode, there will still be one or two people in the car monitoring the autonomous driving, whom might be susceptible to great risks if the brakes do not work. To keep the safety of the vehicle at acceptable levels there is a manual brake pedal that overrides the autonomous drive mode, and brakes the vehicle if necessary. This brake pedal is controlled by the person in the drivers seat, who has experience with the vehicle and is much aware of the risks and behavior of the car. 

The vehicle is not legal to drive in regular traffic, and thus is only driven in big closed off areas, such as a rented runway on Arlanda or in a closed off parking lot, where it is driven at low speeds of up to 45 km/h. The vehicle is also equipped with seat belts dimensioned for racing. 
It is an electric car so the people around wont be exposed to any harmful exhausts. 


\section{Risk assessment}
The risks that is involved with this project include the availability of the \gls{rcv} as well as that ordered hardware gets here on time. Concerning the availability of the \gls{rcv} is mainly concerning test. I do not know how to drive the vehicle or how the whole system works, so I will need help doing the tests. Since the testing phase and later parts of the thesis takes place during summer, I might need to change my plan to match peoples vacation. \newline

Concerning the hardware, the actuators that are mounted on the vehicle is relatively slow, and the system will need to be upgraded to achieve a fast enough response time. The lead times in deliveries must be taken into considerations in the time plan.

\section{Requirements?}

- Speed of the actuators, in ms before request to actual torque meet.

- Should it be optimized for torque meet, or regeneration? that is, should the regenerative me used as torquefill or as main brake? Probably torque fill.

- Maximum negative acceleration/ deceleration.

\section{Outline}
What does the report discuss?
This chapter discusses this, this other chapter discusses that...


%%%%%%%%%%%%%%%%%%%%%%%%%%%%%%%%%

\chapter{Background study}
This section presents the background study
\section{Frame of reference}
The literature search was done directly in the IEEE Xplore database as well as Google Scholar. The search began with broad definitions as 'braking system autonomous driving' as well as 'braking control methods in autonomous driving', which resulted in a few methods that appeared frequently. These methods where then included in the search, together with searches solely for that method.


\subsection{Hardware setup}
Yu et al. \cite{Yu} discusses differences in a linearly actuated system vs a hydraulic pump system. The linearly actuated system uses a linear actuator that press directly on a hydraulic cylinder, while a hydraulic pump system uses an electric pump to build up pressure and controls the pressure in the system by valves and/or solenoids. The linear actuators are simpler and more fail safe, since it doesn't have as many valves from where there can be a leakage of hydraulic fluid. 


\vspace{5mm}
Line, Manzie and Good \cite{4475522} has constructed a electro-mechanical  brake-by-wire system, which utilizes an electrical motor in the calipers as actuator that controls the pressure between the brake pads and the rotor. There is also a brake pedal that sense the pedal position which in turn sends the brake request to a controller. They compare two different controllers in this paper, a cascaded \gls{pi} controller and a \gls{mpc}. The cascaded \gls{pi} controller has three control loops for force, motor angular velocity and motor current. The \gls{pi} controller works, but due to the systems nonlinearity it is somewhat inconsistent in different situations. The \gls{mpc} performs better in this case, but in order for it to work this efficiently a very good model of the system is needed as ground work, and this might be time consuming. \newline


Xiang et al. \cite{Xiang} writes that a electro-mechanical system is preferable over a electro-hydraulical system, due to the simplicity, the efficiency and stability, the enhanced diagnostic capabilities, cost reduction, space and weight saving as well as the elimination of environmental concerns associated with traditional hydraulic braking systems.\newline

%\vspace{5mm}
Line, Manzie and Good \cite{2004-01-2050} as well as in Ahn et al. \cite{ahn2009analysis} is articles about using a cascaded \gls{pi} controller to control a electro-mechanical brake-by-wire system. Here they present requirements on a electromechanical braking system. They discuss the influence of friction, which makes the system nonlinear. The nonlinearity is discussed in the conclusion, where the nonlinear system is the explanation why the cascaded pi controller does not work as fast in the lower brake pad force spectrum as it does for the higher part of the spectrum. \newline



Frede, Khodabakhshian and Malmquist \cite{Frede460614} has done a state-of-the-art report on by-wire systems, with an extensive part about brake-by-wire. They present an overview of brake blending strategies as well as control strategies to regulate braking torque. Most reports discuss brake blending control, but reports where  brake torque control is achieved by fuzzy logic is presented as well. Isermann \cite{661149} also describes the strengths of a fuzzy logic controller, due to its ability to handle nonlinear systems. \newline

Milan{\'e}s et al. \cite{milanes2010electro} presents in an article how an autonomous braking system is implemented into a ordinary road car. The car is already fitted with a hydraulic braking system with a manually controlled braking pedal, and the autonomous braking system is added on to that, resulting in a electro-hydraulic autonomous braking system similar to that on the \gls{rcv}. 
Although the actuator in this car is a electric pump compared to a linear actuator on the \gls{rcv}, this report shows that a electro-hydraulic system is satisfactory, even though other reports state that electro-mechanical systems are preferable \cite{MechatronicsBook} \cite{Xiang}. \newline

The brake blending will be done by a simple function where the regenerative brakes brakes as much as possible, and when they have reached their maximum braking power, the friction brakes steps in to fill in the missing torque, as described by Troung \cite{truongdevelopment}. \newline


\section{Results/Conclusions from background study}
The results from the literature study is that the two friction brake controllers will consist of a \gls{pid} controller and a fuzzy logic controller. 
The brake blending algorithm will be a simple one where the regenerative brakes brake as much as possible, and the friction brakes fills in the missing torque. 


%%%%%%%%%%%%%%%%%%%%%%%%%%%%%%%%%

\chapter{Implementation?}

This chapter could present how the brake is implemented, this in terms of how it is set up in simulink/simscape and how the new mounts was made together with why the decision was made to order new ones.

\section{New actuators}

\section{Hardware for new actuators}

\section{Implementation and tuning in simulink/simscape}



%%%%%%%%%%%%%%%%%%%%%%%%%%%%%%%%%

\chapter{Results}
This chapter presents the results of the thesis.
\section{Hardware}
The highest needed braking torque on each wheel was calculated with respect to the maximum negative acceleration that is stated in the requirements. If the acceleration is known, as well as the mass of the vehicle the total force that needs to act on the vehicle can be calculated with Newtons second law, 
\begin{equation}
F_{tot}=m\cdot a,
\end{equation}
where $F_{tot}$ is the total force needed to achieve acceptable deceleration, $m$ is the mass of the vehicle and $a$ is the acceleration. The vehicle has four wheels and the force is considered to be divided evenly distributed on each wheel. Hence, this gives us 
\begin{equation}
F_{wheel}=\frac{F_{tot}}{4},
\end{equation}
where $F_{wheel}$ is the force needed on each wheel. The torque needed on that wheel can then be calculated by 
\begin{equation}
M=F_{wheel}\cdot r,
\end{equation}
where $M$ is the torque and $r$ is the radius of the wheel. The required torque on each wheel was calculated to 350 Nm.
%%%%%%%%%%%%%%%%%%%%%%%%%%%%%%%%%

\chapter{Conclusion}
This is what a chapter looks like.
%%%%%%%%%%%%%%%%%%%%%%%%%%%%%%%%%

\chapter{Discussion}
This is what a chapter looks like.
%%%%%%%%%%%%%%%%%%%%%%%%%%%%%%%%%

\chapter{Future work}
This is what a chapter looks like.
\\
\glsaddall
\printglossary
\printglossary[type=\acronymtype]

%%%%%%%%%%%%%%%%%%%%%%%%%%%%%%%%%


\bibliography{mybib}
\appendix
\addappheadtotoc
\chapter{First Appendix}

Appendix text goes here.
\glsaddall
\end{document}
